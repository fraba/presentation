% To read acronym file
\usepackage[nolist]{acronym}

\setbeamertemplate{sidebar right}{}
\setbeamertemplate{footline}{%
\hfill\usebeamertemplate***{navigation symbols}
\hspace{1cm}\insertframenumber{}/\inserttotalframenumber}

\renewcommand\rmdefault{pplx}
\renewcommand\mathfamilydefault{cmr}
\usepackage{adjustbox}

% Number style
\usepackage[osf,sc]{mathpazo}

% Reference to list items
% \usepackage[loadonly]{enumitem} % <- Incompatible

% Comment out section by \begin{comment} & \end{comment}
\usepackage{comment}

%Reference name instead of number
\usepackage{nameref}

% Figures, letter spacing and caption style
\usepackage{microtype}
\usepackage[font={small,it},labelfont=bf]{caption}
\usepackage[font={scriptsize,it}]{subcaption}
\captionsetup{compatibility=false}
% \usepackage[font={rm,md,up},margin=10pt,singlelinecheck=false]{subfig}
\usepackage{tikz}
\usetikzlibrary{arrows, positioning, shapes.geometric, shadows, backgrounds, decorations.text}
\usepackage{soul}

% Font encoding. Among others, solve problems with accented chars in output
\usepackage[T1]{fontenc}

% Paragraph spacing
\usepackage{parskip}

% For tables
\usepackage{pgfplotstable}
\usepackage{booktabs}
\usepackage{array}
\usepackage{etoolbox} 
\usepackage{dcolumn}

% \newcolumntype{C}{>{\centering\arraybackslash}m{2.5em}}
% \newcolumntype{D}{>{\arraybackslash}m{15em}}
% \newcolumntype{E}{>{\centering\arraybackslash}m{3.4em}}
% \newcolumntype{F}{>{\centering\arraybackslash}m{4.4em}}

\pgfplotstableset{
    %font={\small},
    empty cells with={--}, %  replace empty cells with ’--’
    every head row/.style={before row=\toprule,after row=\midrule},
    every last row/.style={after row=\bottomrule}
}

%Separate digits with comma (e.g. 1,000,000)
\usepackage[group-separator={,}]{siunitx}
\sisetup{
  detect-all,
  detect-inline-family=math,
  detect-inline-weight=math,
  detect-display-math=true,
  round-mode=places,
  round-precision=2,
  round-minimum = 0.01}

\usepackage{amsmath}

% Set up the images/graphics package
\usepackage{graphicx}
\setkeys{Gin}{width=\linewidth,totalheight=\textheight,keepaspectratio}
\graphicspath{{graphics/}}
\newsavebox\mybox % <- To save size of a picture

% After table/figure space with no idennt
\def\changebreak{\par\vspace{\baselineskip}\noindent}

% The units package provides nice, non-stacked fractions and better spacing
% for units.
\usepackage{units}

% The fancyvrb package lets us customize the formatting of verbatim
% environments.  We use a slightly smaller font.
\usepackage{fancyvrb}
\fvset{fontsize=\normalsize}

% Small sections of multiple columns
\usepackage{multicol}

% Provides paragraphs of dummy text
\usepackage{kantlipsum}

% Figure size
\newlength\onecolumnimage
\setlength\onecolumnimage{.45\textwidth}
\newlength\halfcolumnimage
\setlength\halfcolumnimage{.25\textwidth}

% % Adds a bookmark and table-of-contents line as if it were a subsection

% \AtBeginSection{\frame{\tableofcontents[currentsection]}}

% \usepackage{bookmark}
% \usepackage{etoolbox}
% \makeatletter
% % save the current definition of \beamer@@frametitle
% \let\nobookmarkbeamer@@frametitle\beamer@@frametitle
% % then patch it to do the bookmarks and/or TOC entries
% \apptocmd{\beamer@@frametitle}{%
%   % keep this to add the frame title to the TOC at the "subsection level"
%   % \addtocontents{toc}{\protect\beamer@subsectionintoc{\the\c@section}{0}{#1}{\the\c@page}{\the\c@part}%
%   %     {\the\beamer@tocsectionnumber}}%
%   % keep this line to add a bookmark that shows up in the PDF TOC at the subsection level
%   \bookmark[page=\the\c@page,level=3]{#1}%
%   }%
%   {\message{** patching of \string\beamer@@frametitle succeeded **}}%
%   {\errmessage{** patching of \string\beamer@@frametitle failed **}}%

% \pretocmd{\beamer@checknoslide}{%
%   % ensure the bookmark is not created if the slide is filtered out
%   \let\beamer@@frametitle\nobookmarkbeamer@@frametitle
%   }%
%   {\message{** patching of \string\beamer@checknoslide succeeded **}}%
%   {\errmessage{** patching of \string\beamer@checknoslide failed **}}%

%  \makeatother

%Create Hyperlinks
% To be loaded after everything (except: cleveref, amsrefs, float before hyperref before algorithm, chappg, sidecap, linguex)
\usepackage{hyperref}
\hypersetup{
  pdfinfo={
    Title={\doctitle},
    Author={\docauthor},
    Subject={\docsubject},
    Keywords={\dockeywords},
    CreationDate={\doccreationdate},
    Creator={Typeset with LaTeX}
  },
  linktoc=all
  % colorlinks,
  % citecolor=blue,
  % filecolor=blue,
  % linkcolor=blue,
  % urlcolor=blue
}
\usepackage{url}